\documentclass[12pt,reqno]{article}
\usepackage[utf8]{inputenc}
\usepackage{amsmath}
\usepackage{amsfonts}
\usepackage{amssymb}
\usepackage{amsthm}	
\usepackage{cases}
\usepackage{enumerate}
\usepackage{color}
\newtheorem{theorem}{Theorem}
\newtheorem{lemma}[theorem]{Lemma}
\newtheorem{proposition}[theorem]{Proposition}
\newtheorem{conjecture}[theorem]{Conjecture}
\newtheorem{example}{Example}
%\theoremstyle{definition}
\newtheorem{definition}[theorem]{Definition}
\newtheorem{remark}{Remark}


% %frequently
\newcommand{\D}{\mathrm{d}}
\newcommand{\Real}{\mathbb{R}}
\newcommand{\cXomplex}{\mathbb{C}}
\newcommand{\Int}{\mathrm{Int}}
\newcommand{\Range}{\mathcal{R}}
\newcommand{\Ker}{\mathcal{N}}
\newcommand{\oOmega}{\overline{\Omega}}


% % % %mathcal
\newcommand{\cA}{\mathcal{A}}
\newcommand{\cB}{\mathcal{B}}
\newcommand{\cC}{\mathcal{C}}
\newcommand{\cD}{\mathcal{D}}
\newcommand{\cF}{\mathcal{F}}
\newcommand{\cG}{\mathcal{G}}
\newcommand{\cL}{\mathcal{L}}
\newcommand{\cQ}{\mathcal{Q}}
\newcommand{\cK}{\mathcal{K}}
\newcommand{\cE}{\mathcal{E}}
\newcommand{\cX}{\mathcal{X}}
\newcommand{\cY}{\mathcal{Y}}
\newcommand{\cM}{\mathcal{M}}
% % %math bb
\newcommand{\bF}{\mathbb{F}}
\newcommand{\bQ}{\mathbb{Q}}
\newcommand{\bK}{\mathbb{K}}
\newcommand{\bG}{\mathbb{G}}
\newcommand{\bJ}{\mathbb{J}}
\newcommand{\bL}{\mathbb{L}}
\newcommand{\bX}{\mathbb{X}}
\newcommand{\bY}{\mathbb{Y}}
% % %math tilde
\newcommand{\tX}{\tilde{X}}
\newcommand{\tY}{\tilde{Y}}
% % %special
\newcommand{\htau}{\hat{\tau}}
\newcommand{\RL}{\mathfrak{L}}
\newcommand{\R}{\mathcal{R}_0}
\newcommand{\subX}{\tilde{X}}

\newcommand{\bmu}{\bm{u}}
\newcommand{\bmv}{\bm{v}}
\newcommand{\bmw}{\bm{w}}

\newcommand{\blue}{\color{blue}}
\begin{document}
\title{A summary for the program of $\R$
%\thanks{The work is partially supported by PRC grant NSFC}
\date{\empty}
\author{Lei Zhang}
}\maketitle
\begin{abstract}
This note is to summary how to compute the basic reproduction ratios numerically.
\end{abstract}
\section{Introduction}
How to calculate $\R$ numerically is a very important problem.
For the autonomous models of ODEs, Van den Driessche and Watmough \cite{van2002reproduction} found that the principal eigenvalue of the next generation matrix admits $\R$. For time-periodic compartmental ODEs models, Wang and Zhao \cite{wang2008threshold} proposed a numerical method to compute $\R$ by solving the unique root of a critical equation(see \cite[Theorem 2.1]{wang2008threshold}). We call it Root method. An alternative method to deal with the problem goes to Posny and Wang \cite{posny2014computing}. They transformed the problem into a matrix eigenvalue problem. For the infinite-dimensional cases, motivated by \cite{van2002reproduction}, Wang and Zhao presented a numerical method to approximate the principal eigenvalue of the next generation operator for the autonomous models of reaction-diffusion equations. Combined with the Root method and the principal eigenvalue of positive operators, a numerical method was developed by Liang, Zhang and Zhao \cite{liang2017lyme}. This method can be applied to various kinds of periodic models, including ODEs, reaction-diffusion equations, nonlocal dispersal equations with or without time delay.  In \cite{yang2018remarks}, Yang and Zhang proposed a direct method to compute $\R$ by computing the spectral radius of the corresponding operator, which can be regarded as a generalized method in \cite{posny2014computing}. We also refer to \cite{diekmann1990definition,bacaer2006epidemic}.

The following part of this note is to describe the algorithm to compute $\R$. In the next section, we show a generalized Power Method to compute the spectral radius of a positive operator. In section \ref{sec:autonomous}, we show a numerical method to calculate $\R$ by approximating to the spectral radius of the associated operator for autonomous cases. In section \ref{sec:periodic}, we show the Root method to compute $\R$. In section \ref{sec:example}, we give ten examples to show how to compute $\R$ in Matlab. In section \ref{sec:summary}, we present a short summary for these methods.
\section{Compute the spectral radius}\label{sec:spectral}
 Now we introduce a generalized Power Method to compute the spectral radius of a positive operator $L$ numerically.

\begin{lemma}\label{lem:com:r(L):1}
	Assume that $(E,E_+)$ is an ordered Banach space with $E_+$ being normal and $\Int (E_+) \neq \emptyset$. Let $L$ be a positive bounded linear operator. Then $r(L)= \lim_{n \rightarrow +\infty } \Vert L^n e \Vert_E^{\frac{1}{n}},~\forall e \in \Int (E_+)$.
\end{lemma}
\begin{lemma}\label{lem:com:r(L):2}
	Assume that $(E,E_+)$ is an ordered Banach space with $E_+$ being normal and $\Int (E_+) \neq \emptyset$, which is equipped with the norm $\Vert \cdot \Vert_E$. Let $L$ be a positive bounded linear operator. Choose $v_0 \in \Int (E_+)$ and define
	$
	a_n=\Vert L v_{n-1} \Vert_E,~
	v_n =\frac{ L v_{n-1}}{a_n}, ~
	~\forall n\geq 1.
	$
	If $\lim\limits_{ n\rightarrow +\infty} a_n$ exists, then $r(L)=\lim\limits_{ n\rightarrow +\infty} a_n$.
\end{lemma}

\begin{remark}\label{rem:com:r(L)}
	The algorithm to compute $r(L)$:
	\begin{enumerate}
		\item $v_0=1$.
		\item $v_n = \frac{L v_{n-1}}{\Vert L v_{n-1} \Vert }$, $\forall n \geq 1$.
		\item Let $a_n =\Vert L v_{n-1} \Vert $, $n\geq 1$
		\begin{enumerate}
			\item By Lemma \ref{lem:com:r(L):1}, $r(L)= \lim_{n \rightarrow +\infty } (\Pi_{k=1}^n a_k )^{\frac{1}{n}}$. This is because
			\begin{equation}
			{\|L^n v_0\|} = \Pi_{k=1}^n a_k .
			\end{equation}
			\item In particular, if $\lim\limits_{ n\rightarrow +\infty} a_n$ exists, then $r(L)=\lim\limits_{ n\rightarrow +\infty} a_n$.
		\end{enumerate}
	\end{enumerate}
\end{remark}

\begin{remark}
	For a  non-negative matrix $L$, it is known that
	\begin{enumerate}
		\item If $L^{k_0}$ is strongly positive for some $k_0>0$, then $a_n$ in Lemma \ref{lem:com:r(L):2} converges.
		\item If $L$ is  irreducible, then there exists some $k_0$ such that $(L+I)^{k_0}$ is positive.
	\end{enumerate}
\end{remark}

\section{Autonomous Cases}\label{sec:autonomous}

Now, we present a numerical method to compute $\R$ for a autonomous model. We refer to \cite{van2002reproduction,wang2012basic}.

Let $X$ be a Banach space with the positive cone $X_+$. Assume that
\begin{enumerate}[(H1)]
\item $F$ is a bounded positive operator on $X$.
 \item  $-V$ is a resolvent positive operator on $X$ and $s(-V)<0$.
\end{enumerate}


Consider the equation
\begin{equation}
	\frac{\D u}{\D t} = (F-V)u
\end{equation}


So $\cL$, $\R$ can be defined by
\begin{equation}
\cL u = FV^{-1} u,~ \R= r(\cL).
\end{equation}
\begin{remark}
	\begin{enumerate}
	\item	It is easy to see that
	$$
	r(FV^{-1})=r(V^{-1}F).
	$$
	\item In Matlab, eigenvalues of an matrix can be computed directly. 
	\item We can also compute the spectral radius of $r(\cL)$ by Remark \ref{rem:com:r(L)} with $L= \cL$.
	 \item Operators V and F can be approximated by discretization.
 	\end{enumerate}
\end{remark}


\section{Periodic cases}\label{sec:periodic}
Next, we introduce the Root method. This method is proposed in \cite{wang2008threshold} and developed in \cite{liang2017lyme}.

\subsection{Periodic cases without time-delay}
Let $X$ be a Banach space with the positive cone $X_+$, and
$$
\bX = \{ u \in C(\Real,X): u (t)= u(t+T)\}
$$
with the maximum norm and the positive cone
$$
\bX_+ = \{ u \in C(\Real,X_+): u (t)= u(t+T)\}
$$
Assume that
\begin{enumerate}[(H1)]
	\item $F(t)$ is positive on $X$ and $T$-periodic for all $t \in \Real$.
	\item Let $\{ \Phi(t,s): t \geq s \} $ be the $T$-periodic evolution family of
	\begin{equation}\label{equ:periodic:internal}
	\frac{\D u(t)}{\D t} = - V(t) u(t),~ t\geq 0.
	\end{equation}
	on $X$.	$\Phi(t,s)$ is positive on $X$ for all $t \geq s$ and $\omega(\Phi)<0$.
\end{enumerate}

Consider the equation
\begin{equation}\label{equ:periodic}
\frac{\D u(t)}{\D t} = F(t) u(t) - V(t) u(t),~ t\geq 0.
\end{equation}



So $\cL$, $\R$ can be defined by
\begin{equation}
\cL u (t) =\int_0^{+\infty}\Phi(t,t-s) F(t-s) u(t-s) \D s,~u \in \bX,
\end{equation}
and
$$\R= r(\cL).
$$
Let $\{ U(t,s;\mu): t\geq s \}$ be evolution family on $X$ of the following system
\begin{equation}\label{equ:periodic:mu}
\frac{\D u(t)}{\D t} = \frac{1}{\mu} F(t) u(t) - V(t) u(t),~ t\geq 0.
\end{equation}
\begin{proposition}\label{prop:th:solution}
	Under some conditions, $\mu = \R$ is the unique solution of $r(U(T,0;\mu))=1$.
\end{proposition}
 We also compute the spectral radius of $r(U(T,0;\mu))$ by Remark \ref{rem:com:r(L)} with $L= U(T,0;\mu)$ and search the solution of the equation $r(U(T,0;\mu))=1$ by the bisection method.

\begin{remark}\label{rem:com:r(U)}
	The algorithm to compute $r(L)$:
	\begin{enumerate}
		\item $v_0=1$ is the initial data.
		\item $v_n = \frac{L v_{n-1}}{\Vert L v_{n-1} \Vert }$, $\forall n \geq 1$. Here $L v_{n-1}=U(T,0;\mu) v_{n-1}$ can be numerically computed by standard numerical method of differential equations.
		\item Let $a_n =\Vert L v_{n-1} \Vert $, $n\geq 1$. Usually, $U(T,0,\mu)$ is eventually positive. Thus, $\lim\limits_{ n\rightarrow +\infty} a_n$ exists, and $r(L)=\lim\limits_{ n\rightarrow +\infty} a_n$.
	\end{enumerate}
\end{remark}
\subsection{Periodic cases with time-delay}
 Let $X$ be a Banach space with the positive cone $X_+$, and
 $$
 \bX = \{ u \in C(\Real,X): u (t)= u(t+T)\}
 ,~
 \cX = C([-\tau,0],X)
 $$
 with the maximum norm and the positive cone
 $$
 \bX_+ = \{ u \in C(\Real,X_+): u (t)= u(t+T)\}
 ,~
 \cX_+ = C([-\tau,0],X_+).
 $$
 Assume that
 \begin{enumerate}[(H1)]
 	\item $F(t)$ is positive from $\cX$ to $X$ and $T$-periodic for all $t \in \Real$.
 	\item Let $\{ \Phi(t,s): t \geq s \} $ be the $T$-periodic evolution family of
 	\begin{equation}\label{equ:periodic:internal:delay}
 	\frac{\D u(t)}{\D t} = - V(t) u(t),~ t\geq 0.
 	\end{equation}
 	on $X$.	$\Phi(t,s)$ is positive on $X$ for all $t \geq s$  and $\omega(\Phi)<0$.
 \end{enumerate}
 Consider the equation
 \begin{equation}\label{equ:periodic:delay}
 \frac{\D u(t)}{\D t} = F(t) u_t - V(t) u(t),~ t\geq 0.
 \end{equation}


 So $\cL$, $\R$ can be defined by
 \begin{equation}
 \cL u (t) = \int_{0}^{+\infty} \Phi(t,t-s) F(t-s) u_{t-s} \D s, u \in \bX,
 \end{equation}
 and
 $$
\R= r(\cL).
 $$
 Let $\{ U(t,s;\mu): t\geq s \}$ be evolution family on $\cX$ of the following system
 \begin{equation}\label{equ:periodic:mu:delay}
 \frac{\D u(t)}{\D t} = \frac{1}{\mu} F(t) u_t - V(t) u(t),~ t\geq 0.
 \end{equation}
 \begin{proposition}\label{prop:th:solution:delay}
 	Under some conditions, $\mu = \R$ is the unique solution of $r(U(T,0;\mu))=1$.
 \end{proposition}
 We also compute the spectral radius of $r(U(T,0;\mu))$ by Remark \ref{rem:com:r(L)} with $L= U(T,0;\mu)$ and search the unique solution of the equation $r(U(T,0;\mu))=1$ by the bisection method.


\section{Example}\label{sec:example}
Next, we give some examples for the above methods. For convenience, we omit the corresponding domain when we give the definition of the operator $V(t)$ and $F(t)$.
\begin{example}[Autonomous, ODEs, 3D]
	Let
	$$
	V:=
	\left(
	\begin{matrix}
	1& 0& 0\\
	0& 2& 0\\
	0& 0& 3
	\end{matrix}
	\right),
	F:=
	\left(
	\begin{matrix}
	0& 0& 1\\
	1& 0& 0\\
	0& 1& 0
	\end{matrix}
	\right).
	$$
	 Then
	 $$
	 \cL=FV^{-1}=
	 \left(
	 \begin{matrix}
	 0& 0& 1\\
	 \frac{1}{3}& 0& 0\\
	 0& \frac{1}{2}& 0
	 \end{matrix}
	 \right),~
	 \R=r(\cL)= (\frac{1}{6})^{\frac{1}{3}}.
	 $$
\end{example}

\begin{example}[Periodic, ODEs, 1D] Let $V(t):=m(t)$, $F(t):=f(t)$, where $T=12$ and
	$$m(t)=0.2(1+0.2\cos(2\pi \frac{t}{T})), ~f(t)=0.35(1+0.2\sin(2\pi \frac{t}{T})).$$
	In this case, $\R = \frac{\int_{0}^{T} f(t) \D t}{\int_{0}^{T} m(t) \D t}=1.75$.
\end{example}

\begin{example}[Periodic, ODEs, 2D] Let
	$$V(t):=
	\left(
	\begin{matrix}
	m_1 (t) & 0\\
	0& m_2(t)
	\end{matrix}
	\right), ~
	F(t):=
	\left(
	\begin{matrix}
	0 & f_{12} (t)\\
	f_{21} (t)& 0
	\end{matrix}
	\right),
	$$ where $T=12$ and
	$$
\begin{aligned}
m_1(t)= 0.2(1+0.8\cos(2\pi \frac{t}{T})),~
m_2(t)= 0.3(1+0.8\cos(2\pi \frac{t}{T})),\\
f_{12}(t)=0.35(1+0.8\cos(2\pi \frac{t}{T})),~
f_{21}(t)=0.5(1+0.8\sin(2\pi \frac{t}{T})).
\end{aligned}
$$
\end{example}

\begin{example}[Periodic, DDEs, 2D] Let
	$$V(t) u(t):=
	\left(
	\begin{matrix}
	m_1 (t) u_1(t) \\
	m_2(t) u_2(t)
	\end{matrix}
	\right),
	F(t) u_t:=
	\left(
	\begin{matrix}
	f_{12} (t) u_2(t-\tau_2)\\
	f_{21} (t) u_1(t-\tau_1)
	\end{matrix}
	\right),
	$$ where $T=12$, $\tau_1=3$, $\tau_2=2$ and
	$$
\begin{aligned}
m_1(t)= 0.2(1+0.8\cos(2\pi \frac{t}{T})),~
m_2(t)= 0.3(1+0.8\cos(2\pi \frac{t}{T})),\\
f_{12}(t)=0.35(1+0.8\cos(2\pi \frac{t}{T})),~
f_{21}(t)=0.5(1+0.8\sin(2\pi \frac{t}{T})).
\end{aligned}
$$
\end{example}


\begin{example}[Periodic, DDEs with periodic delay, 2D] Let
	$$V(t) u(t):=
	\left(
	\begin{matrix}
	m_1 (t) u_1(t) \\
	m_2(t) u_2(t)
	\end{matrix}
	\right),
	F(t) u_t:=
	\left(
	\begin{matrix}
	f_{12} (t) u_2(t-\tau_2(t))\\
	f_{21} (t) u_1(t-\tau_1(t))
	\end{matrix}
	\right),
	$$ where $T=12$, $\tau_1=1.8\cos(2\pi \frac{t}{T})+2$, $\tau_2=1.8\sin(2\pi \frac{t}{T})+2$ and
	$$
	\begin{aligned}
	m_1(t)= 0.2(1+0.8\cos(2\pi \frac{t}{T})),~
	m_2(t)= 0.3(1+0.8\cos(2\pi \frac{t}{T})),\\
	f_{12}(t)=0.35(1+0.8\cos(2\pi \frac{t}{T})),~
	f_{21}(t)=0.5(1+0.8\sin(2\pi \frac{t}{T})).
	\end{aligned}
	$$
\end{example}

\begin{example}[Autonomous, PDEs, scalar equation] Let $\oOmega=[0,1]$,
	$$[F u](x) := f(x) u(x),~ [V u](x) := -[ d \Delta u(x) - m(x) u(x)]$$ with Neumann boundary condition, $$m(x)=1+0.5\cos(\frac{\pi}{2} x),~ f(x)=1 +0.5 \sin(\frac{\pi}{2} x)$$ and $d=0.01$.
	Then
	$$
	\cL=FV^{-1},~
	\R=r(\cL).
	$$
\end{example}


\begin{example}[Autonomous, PDEs, two equations] Let $\oOmega=[0,1]$,
	$$
	[F u](x) :=
	\left(
	\begin{matrix}
	f_{12} (x) u_2(x) \\
	f_{21} (x) u_1(x)
	\end{matrix}
	\right),
	[V u](x) :=-
	\left(
	\begin{matrix}
	d_1 \Delta u_1(x) - m_1(x) u_1(x) \\
	d_2 \Delta u_2(x) - m_2(x) u_2(x)
	\end{matrix}
	\right)$$ with Neumann boundary condition and
	$$f_{12}(x)=1+0.5\cos(\frac{\pi}{2} x),~ f_{21}(x)= 1+ 0.5\sin(\frac{\pi}{2} x),$$ $$m_1(x)=1+0.5\cos(\frac{\pi}{2} x),~ m_2(x)=1+0.5\sin(\frac{\pi}{2} x),$$ and $d_1=0.01$, $d_2=0.02$.
	Then
	$$
	\cL=FV^{-1},~
	\R=r(\cL).
	$$
	In this case, $\R=1$.
\end{example}


\begin{example}[Periodic, PDEs, scalar equation] Let $\oOmega=[0,1]$,
	$$[F(t) u](x) := f(x,t) u(x),$$
	$$[V(t) u](x) :=- [d \Delta u(x) - m(x,t) u(x)],$$ with Neumann boundary condition, $$m(x,t)=(1+0.5\cos(\frac{\pi}{2} x))(1+0.5\cos(2\pi \frac{t}{T})),$$
	$$f(x,t)=(1 +0.5 \sin(\frac{\pi}{2} x))(1+0.5\cos(2\pi \frac{t}{T})),$$  $d=0.01$ and $T=12$.
\end{example}


\begin{example}[Periodic, PDEs, two equations] Let $\oOmega=[0,1]$,
	$$[F(t) u](x) := 	\left(
	\begin{matrix}
	f_{12} (x,t) u_2(x) \\
	f_{21} (x,t) u_1(x)
	\end{matrix}
	\right),$$
	$$[V(t) u](x) :=	
	-\left(
	\begin{matrix}
	d_1 \Delta u_1(x) - m_1(x,t) u_1(x) \\
	d_2 \Delta u_2(x) - m_2(x,t) u_2(x)
	\end{matrix}
	\right)$$ with Neumann boundary condition,
	$$m_1(x,t)=(1+0.5\cos(\frac{\pi}{2} x))(1+0.5\cos(2\pi \frac{t}{T})),$$
	$$m_2(x,t)=(1+0.5\sin(\frac{\pi}{2} x))(1+0.5\cos(2\pi \frac{t}{T})),$$
	$$f_{12}(x,t)=(1 +0.5 \cos(\frac{\pi}{2} x))(1+0.5\sin(2\pi \frac{t}{T})),$$
	$$f_{21}(x,t)=(1 +0.5 \sin(\frac{\pi}{2} x))(1+0.5\sin(2\pi \frac{t}{T})),$$
	$d_1=0.01$, $d_2=0.02$ and $T=12$.
\end{example}

\begin{example}[Periodic, PDEs with time delay, two equations] Let $\oOmega=[0,1]$,
	$$[F(t) \phi](x) := 	\left(
	\begin{matrix}
	f_{12} (x,t) \phi_2(-\tau)(x) \\
	f_{21} (x,t) \phi_1(0)(x)
	\end{matrix}
	\right),$$
	$$[V(t) u](x) :=-	
	\left(
	\begin{matrix}
	d_1 \Delta u_1(x) - m_1(x,t) u_1(x) \\
	d_2 \Delta u_2(x) - m_2(x,t) u_2(x)
	\end{matrix}
	\right)$$ with Neumann boundary condition,
	$$m_1(x,t)=(1+0.5\cos(\frac{\pi}{2} x))(1+0.5\cos(2\pi \frac{t}{T})),$$
	$$m_2(x,t)=(1+0.5\sin(\frac{\pi}{2} x))(1+0.5\cos(2\pi \frac{t}{T})),$$
	$$f_{12}(x,t)=(1 +0.5 \cos(\frac{\pi}{2} x))(1+0.5\sin(2\pi \frac{t}{T})),$$
	$$f_{21}(x,t)=(1 +0.5 \sin(\frac{\pi}{2} x))(1+0.5\sin(2\pi \frac{t}{T})),$$
	$d_1=0.01$, $d_2=0.02$, $T=12$ and $\tau=0.6$.
\end{example}

\section{Summary}\label{sec:summary}

For an autonomous model, $\R$ can be calculated numerically by computing the spectral radius of the corresponding operator. For a time-periodic model, $\R$ can be calculated by transfering to solve another problem, which is called Root method. In \cite{yang2018remarks}, we proposed a direct method to compute $\R$ by computing the spectral radius of the corresponding operator, which can be regarded as a generalized method in \cite{posny2014computing}. But it costs too much memory and it is difficult to run on a personal computer for a complex model. So we don't introduce it in this note.
%\bibliographystyle{/home/dylan/Documents/Project/bb/siamplain}
%\bibliography{/home/dylan/Documents/Project/bb/dd.bib}

\begin{thebibliography}{1}
	
	\bibitem{bacaer2006epidemic}
	{\sc N.~Baca{\"e}r and S.~Guernaoui}, {\em The epidemic threshold of
		vector-borne diseases with seasonality}, J. Math. Biol., 53 (2006),
	pp.~421--436.
	
	\bibitem{diekmann1990definition}
	{\sc O.~Diekmann, J.~Heesterbeek, and J.~A. Metz}, {\em On the definition and
		the computation of the basic reproduction ratio {$R_0$} in models for
		infectious diseases in heterogeneous populations}, J. Math. Biol., 28 (1990),
	pp.~365--382.
	
	\bibitem{liang2017lyme}
	{\sc X.~Liang, L.~Zhang, and X.-Q. Zhao}, {\em Basic reproduction ratios for
		periodic abstract functional differential equations (with application to a
		spatial model for lyme disease)}, J. Dynam. Differential Equations, DOI
	:10.1007/s10884-017-9601-7.
	
	\bibitem{posny2014computing}
	{\sc D.~Posny and J.~Wang}, {\em Computing the basic reproductive numbers for
		epidemiological models in nonhomogeneous environments}, Applied Mathematics
	and Computation, 242 (2014), pp.~473--490.
	
	\bibitem{van2002reproduction}
	{\sc P.~van~den Driessche and J.~Watmough}, {\em Reproduction numbers and
		sub-threshold endemic equilibria for compartmental models of disease
		transmission}, Math. Biosci., 180 (2002), pp.~29--48.
	
	\bibitem{wang2008threshold}
	{\sc W.~Wang and X.-Q. Zhao}, {\em Threshold dynamics for compartmental
		epidemic models in periodic environments}, J. Dynam. Differential Equations,
	20 (2008), pp.~699--717.
	
	\bibitem{wang2012basic}
	{\sc W.~Wang and X.-Q. Zhao}, {\em Basic reproduction numbers for
		reaction-diffusion epidemic models}, SIAM J. Appl. Dyn. Syst., 11 (2012),
	pp.~1652--1673.
	
	\bibitem{yang2018remarks}
	{\sc T.~Yang and L.~Zhang}, {\em Remarks on basic reproduction ratios},
	Submitted,  (2018).
	
\end{thebibliography}
\end{document}
